\documentclass[a4paper, 10pt]{article}

\usepackage[margin = 1in]{geometry} % for spacing around
\usepackage{graphicx} % for including images in your pdfs
\usepackage{xcolor} % for including colors in your pdf
\usepackage{soul} % for text decoration
\usepackage[utf8]{inputenc} % for encoded text
\usepackage[T1]{fontenc}
\usepackage{setspace} % for setting different line spacings between paragrafs.
\usepackage{enumerate} % for letting us get more detailed enumerate lists
\usepackage{multirow} % to let us combine more rows together
\usepackage{colortbl} % for decorating tables
\usepackage{amsmath} % used for representing more complicated math displays
\usepackage{supertabular}
\usepackage{longtable} % both of these packages are used to making really big tables
\usepackage{wrapfig} % allows us to wrap text around figures
\usepackage{fancyhdr} % for making fancy headers
%\usepackage{bibtex} % for making better bibliographies
\usepackage[pdftex]{hyperref} % for letting us make links
\usepackage{lscape} % Allows us to flip from portrait to landspace
\usepackage{tikz} % for high detailed drawing
\usepackage{multicol} % To put things side by side
\usepackage{rotating} % For rotating objects
% \usepackage{draftwatermark} % For adding watermarks
\usepackage{MnSymbol} % for using multiple symbols
\usepackage{mathtools} % Used for more math symbols
\usepackage{xfrac} % For more complciated fractions and to add derivitives
\usepackage{hyperref} % for hyper links
\usepackage{enumitem} % for better enum lists
\usepackage{tcolorbox} % for adding colored text boxes
\usepackage{bm} % Adding bold text to math inputs
\usepackage[framed,numbered,autolinebreaks,useliterate]{mcode} % Used for displaying matlab code 
\usepackage{pgfplots} % Used for plotting functions

% Setting up the default image path
\graphicspath{{./Images/}}

% Implementing authro details
\title{Documentation for the \emph{octave} coffee machine project}
\author{Emre Arapcic-Uevak & Vedad Siljic}
\date{\today}

% Setting up the fancy page style
\fancypagestyle{customStyle}{
	\lhead{} \chead{} \rhead{}
	\lfoot{} \cfoot{\thepage} \rfoot{}
	\renewcommand{\headrulewidth}{0pt}
	\renewcommand{\footrulewidth}{1pt}
}
\pagestyle{customStyle}

% Setting up hyperref options
\hypersetup {
	colorlinks = false,
	citecolor = black,
	filecolor = blue,
	linkcolor = blue,
	urlcolor = blue,
	pdftex
}

% Custom commands
\def\checkmark{\tikz\fill[scale=0.45](0,.35) -- (.25,0) -- (1,.7) -- (.25,.15) -- cycle;} 

\begin{document}
	\begin{figure}
		\center
		\includegraphics[width = .35\textwidth]{IUS_Logo}
	\end{figure}

	
	\maketitle
	\vspace{5mm}
	
	\begin{abstract}
		This is a documentation file for the final project assigment for the course \textbf{ENS101}
	\end{abstract}
	\pagebreak
	
	\tableofcontents
	\pagebreak

	\section{Assignments}
		\noindent The following table will show all the assignemtns taken by the corresponding students:
		{
			\center
			\begin{tabular}{|c|c|c|}
				\hline
					& \textsc{Emre Arapcic-Uevak} & \textsc{Vedad Siljic} \\ \hline
				Design  & \checkmark \checkmark & x \\ \hline
				Functionality & x & \checkmark \checkmark \\ \hline
			\end{tabular}
			\par
		}
		
	\section{Design}
		\noindent Lets take a look at the code that was used to generate all the ui components
		\begin{lstlisting}
 % Set up all needed frames
  mainFrame = figure(
		'position', [0 0 800 600],
		'Name','Coffee Machine',
		'NumberTitle', 'off',
		'color', primaryColor,
		'toolbar','none',
		'resize', 'off');

  titleFrame = uipanel(
		'Parent', mainFrame,
		'position', [0 .85 1 .15],
		'HighLightColor', primaryColor);

  rightHalf = uipanel('Parent', mainFrame,
		'position', [0.5 0 0.5 .85],
		'backgroundcolor', primaryColor,
		'HighLightColor', primaryColor );

  keypadFrame = uipanel('Parent', rightHalf,
		'position', [0.05 0.3 0.5 0.5],
		'HighLightColor', primaryColor );

  coffeeSlotFrame = uipanel('Parent', rightHalf,
		'position', [0.6 0.3 0.35 0.5],
		'HighLightColor', primaryColor );

  leftHalf = uipanel('Parent', mainFrame,
		'position', [0 0 .5 0.85],
		'backgroundcolor', primaryColor ,
		'HighLightColor', primaryColor );
		
  coffeeMenu = uipanel('Parent', leftHalf,
		'position', [0.025 0.05 0.95 .9],
		'backgroundcolor', primaryColor,
		'HighLightColor', primaryColor );

  % Set up all elements
  data.coffeeGifDisplayAxes = axes('Parent', coffeeSlotFrame,
		'position', [0 0 1 1],
		'xtick', [], 'ytick', [], 'xlim', [0 1], 'ylim', [0 1],
		'Color', primaryColor);
  data.consoleOutput = uicontrol(rightHalf, 'Style', 'edit', 'units', 'normalized' ,'string', 'IUS Coffee Machine','position', [0.05 0.85 0.9 0.13], 'Max', 5, 'Min', 0, 'enable', 'off','backgroundcolor', '#000000');

  % Set up coffee animation
  axes(data.coffeeGifDisplayAxes)
  frame = imread('./CoffeePercentBarFrames/finalCoffeeVideo-0.png');
  imshow(frame, []);

  % Set up title
  uicontrol('parent', titleFrame,
		'Style', 'text',
		'units', 'normalized' ,
		'string', 'IUS COFFEE',
		'position', [0 0 1 1],
		'backgroundcolor', primaryColor, 'fontsize', 18);

		\end{lstlisting}

		\noindent Quick note about about some of the functions used in the given code block:
		\begin{itemize}
			\item \textcolor{green}{Figure} is an function used for generating a new figure with the given properties.
			\item \textcolor{green}{Uipanel} is a function used for generating a panel(frame) which acts as a group for other ui components.
			\item \textcolor{green}{Axes} is a function used for generating a new axes with the given properties.
		\end{itemize}
		\noindent As we can see from line 2 to 8 we are generating a new figure that will not have a tool bar and will be called "Coffee Machine", it will as well not be resizable. it will start on the bottom left of the users screen with a fixed size of
		$800 \times 600$ pixels. \\

		\noindent After that from line 10 all the way down to line 36 we will seperate the figure into multiple frames on on the far top for the title and then split it in 2 vertically with the rest of the space remaining, we called these spaces
		\emph{rightHalf} and \emph{leftHalf} in our program. Afterwards we will set up some \emph{Axes} that will parented to the given frames which will be later used for image displaying. The coffeeGifDisplayAxes is an Axes that will be used for displaying a
		progress bar type gif, but unfortunately since \emph{Octave} is really slow at handling image loading and does not have \emph{GIF} \textbf{nor} \emph{Video} support all the images had to be loaded one after another.\\

		\hline
		\begin{lstlisting}
% Set up menubar
  interactions = uimenu('Text', 'Interactions');
  interactionsItem1 = uimenu(interactions, 'Text', 'Insert 5KM', 'callback',
		{ @interactionPressed, '5         ' });

  interactionsItem2 = uimenu(interactions, 'Text', 'Insert 2KM', 'callback',
		{ @interactionPressed, '2         ' });

  interactionsItem3 = uimenu(interactions, 'Text', 'Insert 1KM', 'callback',
		{ @interactionPressed, '1         ' });

  interactionsItem4 = uimenu(interactions, 'Text', 'Insert 50F', 'callback',
		{ @interactionPressed, '0.5       ' });

  interactionsItem5 = uimenu(interactions, 'Text', 'Insert 20F', 'callback',
		{ @interactionPressed, '0.2       ' });

  interactionsItem6 = uimenu(interactions, 'Text', 'Insert 10F', 'callback',
		{ @interactionPressed, '0.1       ' });

  interactionsItem7 = uimenu(interactions, 'Text', 'Insert 5F', 'callback',
		{ @interactionPressed, '0.05      ' });

  interactionsItem8 = uimenu(interactions, 'Text', 'Take the coffee', 'callback',
		{ @interactionPressed, 'takeCoffee' });

  interactionsItem9 = uimenu(interactions, 'Text', 'Take the change', 'callback',
		{ @interactionPressed, 'takeChange' });
		\end{lstlisting}
		\noindent The above presented code is fairly simple to understand, firstly we use \textcolor{green}{Uimenu} function to create a new menu component that will be dispalyed on the top of the frame, afterwards we just create the wanted tabs, parents them
		to the given menu we just created, change their text, and a call back function to every single one of them.

		\pagebreak

		\begin{lstlisting}
% Set up KeyPad Buttons
  for i = 1:4
	  for j = 1:3
		  if (i != 4)
			  uicontrol(keypadFrame,
			  'Style', 'pushbutton',
			  'units', 'normalized' ,
			  'string', num2str(3*(i-1) + j - 1),
			  'callback', { @buttonPressed, num2str(3*(i-1) + j - 1) },
			  'position', [1/3*(j-1) 1-1/4*i 1/3 1/4],
			  'Max', 5, 'Min', 0, 'enable', 'on', 'fontsize', 14, 'backgroundcolor', secondaryColor);
		  else
			  if (j == 2)
				  uicontrol(keypadFrame,
				  'Style', 'pushbutton',
				  'units', 'normalized' ,
				  'string', '9',
				  'callback', { @buttonPressed, '9' },
				  'position', [1/3 0 1/3 1/4],
				  'Max', 5, 'Min', 0, 'enable', 'on',
				  'fontsize', 14,'backgroundcolor', secondaryColor);
			  elseif (j == 1)
				  uicontrol(keypadFrame,
				  'Style', 'pushbutton',
				  'units', 'normalized' ,
				  'string', 'Yes',
				  'callback', { @buttonPressed, 'yes' },
				  'position', [0 0 1/3 1/4],
				  'Max', 5, 'Min', 0, 'enable', 'on',
				  'fontsize', 14,'backgroundcolor', secondaryColor);
			  else
				  uicontrol(keypadFrame,
				  'Style', 'pushbutton',
				  'units', 'normalized' ,
				  'string', 'No',
				  'callback', { @buttonPressed, 'no ' },
				  'position', [2/3 0 1/3 1/4],
				  'Max', 5, 'Min', 0, 'enable', 'on',
				  'fontsize', 14,'backgroundcolor', secondaryColor);
			  end
		  end
	  end
  end
		\end{lstlisting}
		\vspace{3mm}
		\noindent The above mentioned code was used to generate all the buttons that can be found on the kaypad frame, here we can see the use of nested loops to allow us to automatically add all the needed buttons, and we can also see that we have a special
		case if we are dealing with the last row since we do not only want numbers to be displayed there, we also want to display the words "Yes", "No". For the rest of the document the documentation about all of these functions will be linked at the end of the
		document

		\pagebreak
		\begin{lstlisting}
% Set up for coffee menu
  for i = 1:3
	  for j = 1:3
		  if (3*(i-1) + j > length(data.coffeeNames)) break; end
		  coffeeFrame = uipanel('Parent', coffeeMenu,
		  'position', [(j-1)/3 1-i/3 1/3 1/3],
		  'HighLightColor', primaryColor ,
		  'backgroundcolor',primaryColor );

		  axes('Parent', coffeeFrame,
		  'position', [0 0.4 1 0.6],
		  'xtick', [], 'ytick', [], 'xlim', [0 1], 'ylim', [0 1],
		  'Color', primaryColor);

		  imshow(strcat('./Images/', data.coffeeNames{3*(i-1) + j}, '.png'), []);

		  uicontrol(coffeeFrame,
		  'Style', 'text',
		  'units', 'normalized' ,
		  'string', num2str(3*(i-1) + j),
		  'position', [0 0.2 1 0.2],
		  'enable', 'on', 'fontsize', 14,
		  'backgroundcolor',primaryColor );

		  uicontrol(coffeeFrame,
		  'Style', 'text',
		  'units', 'normalized' ,
		  'string', data.coffeeNames{3*(i-1) + j},
		  'position', [0 0 1 0.2],
		  'enable', 'on', 'fontsize', 14,
		  'backgroundcolor',primaryColor );
	  end
  end
		\end{lstlisting}

		\vspace{3mm}
		\noindent The above mentioned code was used to generate all images for coffees that can be found in the coffee menu upon ordering, what the code does is it goes by the same principal as the generation for keypad buttons by using a nested loop to
		make frames that all take $\frac{1}{3}$ of width and height of the available space. Afterwards it makes an axis and 2 text components that is lays next to each other vertically and sets all of their properties. \\
		\hline
		
		\begin{lstlisting}
guidata(mainFrame, data);
		\end{lstlisting}

		\noindent And finally we will use this line of code to set up everything we have saved in the structure variables called \emph{"data"} and link that variable to the figure. We do this because Octaves scope is function based meaning any variables made
		in a function is \emph{only} visible in the said function. But by doing this we can use the function guidata again to read the structure we connected to the figure element.
\end{document}
